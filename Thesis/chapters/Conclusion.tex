In this thesis, the Bayesian inference of hydrological model parameters is implemented. Four versions of Markov chain Monte Carlo algorithms are used for performing Bayesian inference, each delivering different results based on their unique properties. The input parameter is then specified, in which the accuracy and efficiency metrics of different input parameters are benchmarked. For some input parameters, obvious relations between the configurations and the metrics can be found. For others, the configurations do not make a huge difference, or there is a certain configuration that stands out among all the different values of input parameters.

The first algorithm that is used is the fundamental Metropolis-Hastings, where one sample is generated in each iteration and later on, accepted or rejected based on the calculated acceptance rate using the likelihood function and sampling kernel. Therefore, the sampling kernel and the likelihood kernel play important roles, in which they exert a great impact on the acceptance or rejection.

The other three algorithms that are used all utilize the parallel aspect. The second algorithm, parallel Metropolis-Hastings is the parallel version of the fundamental Metropolis-Hastings algorithm, in which it uses multiple chains for sampling instead of one single chain. The number of chains is, therefore, a relevant factor for the result, since the number of samples generated in each chain is closely related to the convergence rate of the final result.

The third algorithm is the general parallel Metropolis-Hastings algorithm, in which multiple samples are generated in each iteration rather than one single sample. The acceptance rates are calculated in the form of a vector, which builds a probability space that allows random sampling to take place. Here, the ratio between the number of samples generated and accepted plays an indispensable role, as discussed in detail in the chapter. The amount of numbers accepted is also an important factor. Other input parameters including sampling and likelihood kernels are also relevant, as they contribute to the calculation of the acceptance rate vector. In comparison to the two algorithms above, this algorithm achieves higher accuracy due to the acceptance mechanism, but potentially lower efficiency due to the complexity of acceptance calculation.

The final algorithm is the DREAM algorithm, which improves the sampling phase of the algorithm by adapting by adjusting proposal distributions based on past samples and employing a crossover mechanism from differential evolution to efficiently explore the parameter space. Thus, the configurations that are related to crossover, DE (differential evolution), and snooker are relevant for the performance of the result. The DREAM algorithm generally delivers the most accurate and efficient result, achieved not only by running multiple parallel chains but also by enhancing convergence and ensuring thorough exploration of the parameter space.

For the general implementation of Markov chain Monte Carlo algorithms, three other factors are relevant, namely the burn-in phase, the effective sample size, and the initial states. The burn-in phase discards the samples generated at the very start because the Markov chain has not entered the stationary phase yet. The effective sample size allows the algorithm to consider only every n-th sample in the result, allowing less dependency to form between samples in the result that are next to each other. A good choice of initial states optimizes the efficiency of the sampling process by allowing the chains to quickly enter the phase where they sample from the stationary distribution.

For the specific use case for the hydrological model, handling the cases where the samples generated are out of bounds is also crucial. Since the prior parameters are uniformly distributed, the out-of-bounds samples could cause odd behaviors. Using mechanisms such as ignoring, reflection, and aggregation, these cases can be well handled.
