Das Verfahren der Bayesschen Inferenz für die Parameter der hydrologischen Modelle spielt eine bedeutende Rolle bei der Ausführung der hydrologischen Modelle, vor allem bezüglich der Genauigkeit und der Zuverlässigkeit. Dieses Thema wird in dieser Arbeit auseinandergesetzt, indem man das Verfahren die Algorithmen zum Markov-Chain-Monte-Carlo-Verfahren implementiert, vor allem mit der Eigenschaft von dem Parallelrechner, damit sich die Genauigkeit und die Effizienz von der Bayesschen Schätzung der Parameter verbessern. Die Ergebnisse bezüglich der Genauigkeit und der Effizienz der Bayesschen Inferenz werden mithilfe der Vergleichsmetriken analysiert, wodurch sie ausführlich visualisiert werden, sodass die Beziehungen zwischen den verschiedenen Implementierungsvarianten und den Ergebnissen klar dargestellt werden können. Außerdem wird die Beziehung zwischen der Wahl der Trainingsdatensätze in Form einer Zeitreihe und die Ergebnisse einer Bayesschen Inferenz beobachtet. Durch die Analyse dieser Aspekte von den spezifischen Implementierungen der Algorithmen und den Datensätzen erwirbt man Kenntnisse über die Leistung und die Ergebnisse der Bayesschen Inferenz, sodass praktische und skalierbare Anwendung bei der hydrologischen Modellierung angewendet werden können.