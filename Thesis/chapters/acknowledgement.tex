I would like to take this opportunity to thank M.Sc. Ivana Jovanovic Buha for her guidance. Her explanations and advice greatly helped me understand the topic of Markov chain Monte Carlo. Our regular meetings and communications motivate me to keep researching this interesting topic. I would also like to thank professor Univ.-Prof. Dr. Hans-Joachim Bungartz for introducing me to numerical computing and scientific computation via lectures and offering me this amazing opportunity to write this thesis in his chair.

I am grateful for my informatics teacher from my high school, Andreas Reisenbauer, from BRG9 Vienna. He exposed me to the exciting field of informatics and allowed me to feel the joy of learning new things, both inside and outside of the field of informatics. His support inspires me to keep working for continuous growth and discovery in my academic career and personal pursuits.

I also have great appreciation for all my friends at the university. With our open communication and their warm-hearted support, I feel extra motivated to work on my thesis. Their constant encouragement and willingness to support me have made these challenging three years of bachelor's enjoyable and rewarding for me.

Last but not least, I would like to thank my whole family and my partner for their continuous support. During the entire process of writing this thesis, I came across a few difficulties in my thesis and life. Their advice helps me a lot to cope with these difficult situations.

