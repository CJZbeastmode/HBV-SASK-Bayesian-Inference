\chapter{Introduction}

The hydrological model HBV-SASK is a computer simulation based on the HBV model, also known as Hydrologiska Byråns Vattenbalansavdelning model. Inheriting the core idea of the HBV model which is used to analyze discharge~\cite{hbv_intro}, it uses the data from Saskatchewan, Canada, as the name SASK suggests. The model has a few hyperparameters representing the hydrological processes specific to the Saskatchewan region~\cite{hydrology}. The uncertainty quantification of these parameters has long been a topic for research, as the precise tuning of these parameters exerts a significant influence on the accuracy and efficiency of hydrological predictions to reflect local water characteristics and climatic conditions~\cite{hbv_calibration_intro}. This is the focus of this thesis, in which different approaches towards the calibration of parameters are used so that the accuracy and the efficiency of the calibration can be explored.

The calibration of these parameters requires a statistic inference method. In this thesis, the Bayesian inference is selected to perform parameters uncertainty quantification, so that the probability distributions of the model parameters can be estimated~\cite{bayesian_uq_intro}. The Markov Chain Monte Carlo (MCMC) method is deployed within the Bayesian framework, being used to sample the posterior from the prior distributions of the parameters. This approach allows for a more comprehensive understanding of the parameter uncertainties, leading to more accurate and reliable hydrological predictions~\cite{mcmc_practice}. The exact process and algorithm are discussed in the thesis in detail.

The Markov Chain Monte Carlo algorithm uses a Markov chain to perform Monte Carlo simulation, repetitively sampling from the parameter space to approximate the posterior distributions. By running multiple chains in parallel, the computational efficiency and convergence rate of the sampling process can potentially be significantly improved. To investigate this aspect, we emphasized the parallel implementation of the Markov chain Monte Carlo algorithm. In this thesis, the following four versions of Markov chain Monte Carlo algorithms are discussed and used:

\begin{itemize}
    \item Metropolis-Hastings: The fundamental Metropolis-Hastings method generates the posterior distribution by proposing a new point based on a chosen proposal distribution and accepting or rejecting it with a probability that ensures detailed balance~\cite{detailed_balance}.
    \item Parallel Metropolis-Hastings: The parallel Metropolis-Hastings is a method that is based on the fundamental Metropolis-Hastings algorithm, but uses multiple Markov chains instead of one single Markov chain to run the algorithm in a parallel way to enhance efficiency.
    \item General Parallel Metropolis-Hastings: The parallel Metropolis-Hastings algorithm extends the fundamental Metropolis-Hastings method by generating multiple samples in one iteration instead of one single sample.
    \item DREAM: The Differential Evolution Adaptive Metropolis (DREAM) algorithm is an advanced MCMC method that combines differential evolution with adaptive Metropolis sampling to generate the posterior distribution. It deploys multiple chains in parallel and proposes new points using the mutual information between different chains so that the proposal distribution is adaptively updated over time~\cite{dream}.
\end{itemize}

The goal of implementing the four different algorithm versions is to observe the optimized accuracy and efficiency performances of the algorithms. This is achieved by exploring configurations of each of these algorithms are explored. For each algorithm, a set of configurations can be tuned to have an impact on the sampling process of the algorithm. These can be anything from the transition kernel, burn-in factors, and initial states, which will be discussed later in detail. By tuning these configurations, results with different accuracy and efficiency metrics are presented, allowing us to make better decisions regarding how to use the algorithm for uncertainty quantification.

For the last part of the thesis, an overview of the selection of data for the training purpose provided alongside the HBV-SASK model is given. A deep look into the different characteristics and anomaly points throughout the entire time series is provided, allowing readers to gain more insights about the provided data itself. Using different periods with different properties for the Markov chain Monte Carlo algorithms to perform training, the differences regarding the accuracy of the inferred result are analyzed.